\chapter{Detailed constitutive models}
\label{ann:A_models}

This appendix provides additional details on the constitutive models
used to describe the in-plane shear behavior of woven structures.
These developments complement the formulations presented in the main
chapters and are provided for completeness.

\section{Kinematic framework}

The deformation of the material is described by the deformation
gradient $\F$. The right Cauchy--Green tensor is defined as
\begin{equation}
	\C = \F^{\mathrm{T}} \F
\end{equation}

For in-plane deformations, the shear component $\gammaxy$ is extracted
from the Green--Lagrange strain tensor.

\section{Shear strain measure}

The in-plane shear strain is defined as
\begin{equation}
	\gammaxy = 2 E_{12}
\end{equation}
where $E_{12}$ denotes the shear component of the Green--Lagrange strain
tensor.

\section{Strain energy decomposition}

The strain energy density function $\W$ is decomposed as
\begin{equation}
	\W = \W_{\text{tension}} + \W_{\text{shear}}
\end{equation}

This additive decomposition allows independent calibration of tensile
and shear responses from dedicated experimental tests.

\section{Shear constitutive law}

A nonlinear shear law is introduced in order to capture the stiffening
observed at large shear angles:
\begin{equation}
	\W_{\text{shear}} = \frac{1}{2} G_{12} \gammaxy^2
	\left( 1 + \alpha \gammaxy^2 \right)
\end{equation}
where $G_{12}$ is the initial shear modulus and $\alpha$ is a
nonlinearity parameter.

\section{Remarks on numerical implementation}

The shear constitutive model is implemented within a \UMAT{} subroutine
in Abaqus. The tangent stiffness matrix is derived analytically to
ensure quadratic convergence of the Newton--Raphson scheme.
